In this project, my role was in the web application team, working closely with Alex, Pierre, and Brieuc to produce the balloon content feed for the client. I also undertook a limited amount of graphics work to produce static images that were used in the client application.

Overall I would say that the project has been very successful and I am pleased with the software we've produced as a team. It's been fun to work with my peers and get to learn from them and likewise share my knowledge.

I feel that there have been points where our group communication could have been better, but despite this we managed to clear up minor misunderstandings and all work towards the same goals, achieving the desired outcomes.

It's been delightful to watch people enjoy playing with the application, and gathering the user feedback was a rewarding part of the project. With seemingly positive feedback, I sincerely hope that this is something that can be further developed in the future.

In terms of our progress, I think we've implemented the foundations of something good, but don't feel that we've necessarily generated a completely stand-alone offering yet. This is because our system incorporates aspects from both a purely functional content delivery concept, and a more interactive, `fun' game concept but it does not fully realise either concept.

I would have liked it if we were able to do more feature iterations because then we could have developed a mini-game perhaps to give the application a greater draw for user interaction, and align it more strongly with a distinct purpose, which is something many users seemed unsure of at first.

In terms of the technical challenges met and tools used, I would not choose the CodeIgniter framework again. Whilst it has proven very useful for getting things running quickly, it has some quirks that seem unintuitive to me regarding naming conventions, and how models are used from the controllers. On balance however, it is probably the best choice of framework for the requirements of this application.

One library I was particularly impressed with was the jQuery Mobile toolkit which was used to make the content rating screens. It was incredibly easy to use, well documented, and consistent across the mobile devices we tested it with.

I would have liked to abstract some of the feed generation a little further (as outlined in the future work section), but whilst prototyping I wasn't planning the architecture to be as extensible as it's now clear would be useful. The iterative development and adjustments to requirements over the weeks also contributed to the slightly under-architected implementation.

Whilst I was involved with the game client and server teams to integrate the feed, it would have been interesting to have a more direct involvement with the C\# and XNA applications. Perhaps if time had permitted extra iterations, I would have been able to pick up a little of the client development.

This project has been a fun and challenging way to conclude my time at Heriot-Watt, and I'm glad that it's been as successful as it has. As mentioned, I hope it's a project that will have further development work carried out on it, and it would ultimately be fantastic to see it running in the crush area at some point in the future.