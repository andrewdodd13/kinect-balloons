I thought that the project was a great success. I have worked on many group project before and they have always been full of problems and last dash rushes to get all the work done by a minority of people. This was nothing like that. It was brilliant. 
The group had a lot of diversity which meant that we had someone who was an expert for every role needed. Some member fitted well into organising and managing the group while others were brilliant coders who knew how to get things to work. Though being programmers, none of us like doing reports\footnote{I don't know about that, I quite like putting reports together! -- Pierre}. 

The management and organisation of the group was really good. I felt that John and Lauren did a good job of taking a lead and organising everyone into their groups. Though people was always willing to get involved and organise themselves as needed. The group was quickly split up into groups of networking, design and management, research, website content submission system and of course the Kinect work. People knew their strengths so when it came to organising the group, people would quickly volunteer for parts of the project they knew they would be good at. For example I have done PHP and web design before so I volunteered for the website. Organisation was helped by the use of Git and Basecamp. These were well used throughout the project and Andrew did a good job of managing the Git Repositories.

Andrew and Williams did a very good job with the Kinect work. They quickly got a prototype up and running to demonstrate that this could actually be done. This was very inspiring and helpful to others as we now knew that what we were trying to achieve was possible and not just going to end up with finding out it doesn't work half way through. Brieuc and Pierre were very good at the networking side of things. They appeared to work well as a team and efficiently solve any problems they came across. Ryan did a good job with research and evaluation. He produced a very good questionnaire for evaluating the Balloon News application with which we managed to get 75 participants for.

I worked with Chris on the web application. This was for submitting content to the website. Together we negotiated to split the work up so that I produced the web front end and he produced the back end that would talk to the screens in JSON. I then produced a website using the CodeIgniter framework which he was easily able to tap into and use for his part. I felt that the back end code was well structured and that it was very efficiently done. However I am disappointed with the design. I have never been a fan of CSS and I do not have a designer's eye, so the overall look of it is not great. However it does work and work well. I found Chris very helpful as he knew more about CSS than I did and I enjoyed working with him. I wish I had been more active with the Kinect application, I got to modify it a little but I never got to get into it properly. Given more time I would have liked to do more of the programming for that.

I felt that the overall application was very good however I did feel that the displaying of content could have been done better. Instead of having the user download their article to the phone, the article should have been displayed inside a web view. I was told that this was infeasible but given more time I think it would have been possible with some tweaking. By having a web view the user would have been able to easily see the article he was viewing and could use his hands to scroll up and down e.g. Move hand to bottom of web view to scroll down. I wanted to put this in as the main negative comment was that people didn't see the point of it, and I thought that the user generated content was not obvious enough to users. More advertising of the submission website was needed. A further idea would be to like the system in with Facebook's `like' system so that users could like something on Facebook and it would appear on the screen. However I imagine this is incredibly hard to do but I though in the future it could be looked into to make submitting content more streamlined.
