This section will report on the overall specification for the application and detail the resources that brought about its construction from the initial client specification to the known background information and current technologies employed. 

\subsection{Initial Concept}
The task set to the development team was to design, implement and evaluate an application for use by both researchers and students within the Mathematical and Computer Science (MACS) department's learning zone that utilized the Microsoft Kinect. The project required the development team to use creative processes in order to develop the application to be fun for users and original in design.

\subsection{Background Information}
The learning zone was purpose built within the waiting area between two of the MACS department's lecture facilities, as a social area for recreational or study use by students. In addition it provided researchers space from which to promote their work and acted as a central zone of information for all involved. It features three open planned seating arrangements in addition to two more enclosed work spaces with three, large TV screens positioned about the adjacent walls, each with their own Kinect sensor. 

\subsection{Current Technologies}
Prior to the start of the project, the learning zone screens each featured an application for displaying various information on content relevant to the department's subject area which ranged from displaying student and researcher work going on in the department, to twitter feeds highlighting headlines on topics of interest.

\subsection{Proposed Solution} 
A solution was proposed which involved displaying these various forms of content in a more central and uniform manner, in addition to serving as an interactive and social application that kept with the recreational aspect of the learning zone. This application would provide a visualization of the data but also a way in which users could interact with each other through the use of separate screens.
