\subsection{Procedure}

To extensively evaluate our balloon application with as many participants as possible, the system was setup in the School of Mathematical and Computer Sciences'  Earl Mountbatten building crush area at Heriot-Watt University from approximately 12.15 p.m. to 2.15 p.m. for five consecutive days, Monday through Friday, in Week 8 of Semester 2.

Before each evaluation session the system was setup using the following equipment:
\begin{itemize}
\item{2 x Microsoft Kinects as the motion sensors used for recognising participants' hand gestures that output video at a frame rate of 30Hz and resolution of 640x480 pixels;}
\item{2 x ?? inch plasma televisions for displaying the balloons and their associated content on screen; }
\item{2 x mid-range laptop computers for the purposes of both to run the devloped balloon client application and one to run the balloon application server that allows balloons to move from one client application to the other.}
\end{itemize}

Every participant invited to be involved in the evaluation session was first presented with a short pre-experiement questionnaire (see appendix \vref{EvalPreQ}) that asked the participant to provide some general information about themselves, including age and gender; if they owned a smartphone that captures Quick Response (QR) codes; if they had any previous experience with motion detection devices like the Kinect; and what they typically use the Earl Mountbatten crush area for.

Upon completing the questionnaire, participants were then taken over to the system where they were allowed to interact with the Kinect by their own free will for however long they chose. Each of the participants were given little to no instruction on how to use the system and were only given subtle hints regarding the capabilites of system if they had difficulties.

Once participants were finished interacting with the system, they were then asked to complete a short post-experiment questionnaire that specifically asked about their ability to use each of the functionalities the system provided, including: pushing balloons, customising balloons and popping balloons to access content; if they felt embarrassed at any point when using gestures to interact with the system; and if they were likely to use the system again in the future.

\subsection{Results}

In total, 75 participants were recruited to help evaluate our system and were predominantely students and members of staff from Heriot-Watt University. Of those, 80.0\% (60) were male and 20.0\% (15) were female; 76.0\% (57) were between 16 and 24 years old, 20\% (15) were between 25 and 33 years old, 2.7\% (2) were between 34 and 42 years old, and 1.3\% (1) were between 43 and 51 years old.

With regard to smartphones, it was shown that of the 75 participants who completed the pre-questionnaire, an overwhelming 77.3\% (58) of particpants stated that they were currently in possession of a smartphone. Following that, a further 55.2\% (32) of smartphone owners went on to indicate that they did in fact have a Quick Response (QR) code reader application installed on their device. From these statistics it shows that a large majority of people nowadays own a smartphone, therefore providing a strong arguement to support the functionality of having QR codes incorporated with the content displayed on screen. And even though some smartphone users may not actually have a QR code reader application installed on their device, they could easily download one in the future if required as it is fair to assume that many of those without one might not have had a justifiable reason to do so at that moment in time.
On the other hand, of the 72 participants who completed the post-questionnaire, 83.3\% (60) of participants admitted to not using a QR code during the time they spent interacting with the system. However, this could perhaps be attributed to participants being unable to pop balloons on screen in order to access content or that participants were pushed for time due to other commitments.

When participants were asked if they had every used any video game related motion detection devices, it was shown that of the 75 participants who had completed the pre-questionnaire, 81.3\% (61) of participants said they did have experience with video game related motion detection devices while 18.7\% (14) said they had not. Of those who responded with having experience with motion detection devices, most stated they were most familiar with using the Nintendo Wii Remote at 88.5\% (54), while quite encouragingly 52.5\% (32) said they had experience with the Microsoft Xbox Kinect and 14.8\% (9) had used the Sony Playstation Move before. Therefore, given by these statistics, it could be hypothesised that participants with prior experience with video game related motion detection devices could easily adapt to interacting with the Kinect balloon application.

This theory can be supported by the results taken from the post-questionnaire where it was shown that 93.0\% (66) of participants were able to intuitively push balloons around screen and 77.8\% (49) of participants were able to intuitively pop a balloon on screen.

An argument to support that the Kinect balloon application should be permanently running on the  plasma screens of the Earl Mountbatten crush area can be derived from the results taken from both the pre-questionnaire and the post-questionnaire. Participants were asked how long they approximately spend in the crush area each day and of the 75 participants, 68\% (51) reponded by stating that they spend less than 20 minutes in the crush area, implying most participants are only in the crush area for a short period of time purposely waiting to attend their next lecture. This can be supported by the reponse given by the question that asked participants to specify what they typically use the Earl Mountbatten crush area for, where 58.7\% (44) of participants mainly said they generally use the crush area as a waiting area. However, given by the results of post-questionnaire, where 81.9\% (59) of participants stated they would like to play with the balloons again in the future and 70.8\% (51) of participants stated they would like to use it again to view more interesting content, perhaps the the Kinect balloon application would prove generally quite popular as something to quickly interact with to read a news article or having fun with between lectures.

Draft Note: generally quite a messy account of the user evaluation sessions made in week 8 thus so far, things to address are as follows: observations made w.r.t. customising balloons, popping balloons, social factors (including embarrassed users and multiplayer use), group study, and finally some general language issues.
