\subsectionauthor{Pilot Study}{Ryan Oswald}

A small pilot study was carried out in Week 7 of Semester 2 with a small group of 10 students from the Advanced Interaction Design course at Heriot-Watt University. Invited participants were given the opportunity to interact with the Kinect balloon application by their own free will while observations would be made based on their behaviour when using the system. From this it was hoped that any qualitative data gathered could perhaps determine what improvements could be made in future iterations of the developed system.

During this session, participants were given a minimal set of tasks in which to perform based on each of the functionalities the Kinect balloon application exhibited. From general observations made, it was shown that all participants were able to push balloons on screen without too much difficulty once they noticed their hands were being tracked on screen via the Kinect. Similarly, it was shown that participants were able customise to balloons on screen, but most of the time this was purely by mistake. It was noted that when asking participants to colour a balloon with a specific colour, it was apparent that they found it quite difficult as it required some precision to manipulate the balloon in such a way that it would not be accidentally coloured by a different paint bucket. When asking each of the participants to pop a balloon on screen, most participants struggled to determine what gesture had to be made in order to pop a balloon on screen. 

However, after each participant had discovered what gesture had to be made, either by accident or by some sort of intervention, it was noted that the task became much more easier to perform by the participant with each successive attempt made. Once participants were able to pop a balloon on screen, it was observed that most participants were unaware of how to close the content that was displayed to them on screen. It was only after some close inspection by participants that it was obvious to close content displayed to them by using the red cross in the top right hand corner of the screen. However, even after noticing that they had to close the content box on screen, participants struggled to perform the correct gesture in order to close the content box. Participants were physically trying to push the button inwards in order to interact with the close button, whereas the correct gesture is to simply hover your hand over the close button for a short time interval.

Other observations made were that some participants were initially embarrassed when using the system as they did not know what gestures to perform in order to interact the balloons on screen and were self-conscious about what gestures to make. Once each of the participants knew what to do and how to interact with the balloons their gestures became much more exaggerated, enabling them to interact with the balloons located on all sides of the screen. Be that as it may, some participants when interacting with the Kinect balloon application in pairs would often get carried away with themselves and impede the other participant when reaching for a balloon on the opposite side of the screen from their position.

\subsectionauthor{Main Study}{Ryan Oswald}

\subsubsection{Procedure}
To extensively evaluate the Kinect balloon application with as many participants as possible and acquire a large amount of quantitative data, the system was setup in the School of Mathematical and Computer Sciences' Earl Mountbatten building crush area at Heriot-Watt University from approximately 12.15 p.m. to 2.15 p.m. for five consecutive days, Monday through Friday, in Week 8 of Semester 2.

Before each evaluation session the system was set up using the following equipment:
\begin{itemize}
\item{2 x Microsoft Kinects as the motion sensors used for recognising participants' hand gestures that output video at a frame rate of 30Hz and resolution of 640x480 pixels;}
\item{2 x 42 inch plasma televisions for displaying the balloons and their associated content on screen;}
\item{2 x mid-range laptop computers for the purposes of two to run the developed balloon client application and one to run the balloon application server that allows balloons to move from one client application to the other.}
\end{itemize}

Every participant invited to be involved in the evaluation session was first presented with a short pre-experiment questionnaire (see appendix \vref{EvalPreQ}) that asked the participant to provide some general information about themselves, including age and gender; if they owned a smartphone that captures Quick Response (QR) codes; if they had any previous experience with motion detection devices like the Kinect; and what they typically use the Earl Mountbatten crush area for.

Upon completing the pre-questionnaire, participants were then taken over to the system where they were allowed to interact with the Kinect by their own free will for however long they chose.

During this time each participant was given little to no instruction on what gestures were required to effectively interact with the Kinect and by extension use the balloon application. However, some participant were given subtle hints that lead them to being able to figure out how to interact with the system if they had any difficulties or if they specifically asked for some assistance.

Once participants were finished interacting with the system, they were then asked to complete a short post-experiment questionnaire that specifically asked about their ability to use each of the functionalities the system provided, including: pushing balloons, customising balloons and popping balloons to access content; if they felt embarrassed at any point when using gestures to interact with the system; and if they were likely to use the system again in the future.

It should be noted that both the pre-experiment questionnaire and the post-experiment questionnaire can be found attached to this document, located in the appendices section.

\subsubsection{Purpose}
The purpose of having this setup and evaluation procedure - where the participant was given both a pre-experiment questionnaire and a post-experiment questionnaire - was to develop conclusions based on the information gathered on the participants themselves in relation to their experience with using the Kinect balloon application.

By asking participants to provide their age bracket in the pre-experiment questionnaire it was hoped that those results in comparison to those from the post-experiment questionnaire, where participants were asked how intuitively they could perform certain gestures, could perhaps help determine whether or not age plays a factor with regards to effectively interacting with the Kinect and motion detection devices in general. Similarly, by asking participants if they had any previous experience with using motion detection devices in the pre-experiment questionnaire it was hoped that those results in comparison to those from the post-experiment questionnaire, where participants were asked how intuitively they could perform certain gestures, could perhaps give an indication of easy the system is to pick up and use for both those who may or may not have used motion detection devices in the past. 

Additionally, by asking participants if they have a smartphone with a QR code reader application installed on it in the pre-experiment questionnaire it was hoped that those results in conjunction with those from the post-experiment questionnaire, where participants were asked if they had used a QR code that was displayed alongside the content they chose to view, could perhaps help justify the inclusion of QR codes in the balloon application. Finally, by asking participants how they typically use the Earl Mountbatten crush area  it was hoped that with those results in conjunction with those from the post-experiment questionnaire, where participants were asked if they were likely to use the balloon application again in the future, could help justify having the balloon application permanently running on the plasma screens in the Earl Mountbatten crush area after project has come to an end.

\subsubsection{Results}

In total, 75 participants were recruited to help evaluate the system and were predominantly students and members of staff from Heriot-Watt University who were passing through or waiting in the Earl Mountbatten crush area at the time. Of those, 80.0\% (60) were male and 20.0\% (15) were female; 76.0\% (57) were between 16 and 24 years old, 20\% (15) were between 25 and 33 years old, 2.7\% (2) were between 34 and 42 years old, and 1.3\% (1) were between 43 and 51 years old respectively.

With regard to smartphones, it was shown that of the 75 participants who completed the pre-questionnaire, 77.3\% (58) of participants stated that they were currently in possession of a smartphone. Following that, a further 55.2\% (32) of smartphone owners went on to indicate that they did in fact have a Quick Response (QR) code reader application installed on their device. From these statistics it shows that a large majority of people nowadays own a smartphone, therefore providing a strong argument to support the functionality of having QR codes incorporated with the content displayed on screen. And even though some smartphone users may not actually have a QR code reader application installed on their device, they can quite easily download one in the future if necessary as it is fair to assume that many of those without such an application may not have previously had a justifiable reason to download one until using the balloon application.

On the other hand, of the 72 participants who completed the post-questionnaire, unfortunately 83.3\% (60) of participants admitted to not using a QR code during the time they spent interacting with the system. However, this perhaps could be attributed to participants being unable to pop balloons on screen in order to access content, being pushed for time due to other commitments or not having their smartphone directly on their person at that time.

When participants were asked if they had every used any video game related motion detection devices, it was shown that of the 75 participants who had completed the pre-questionnaire, 81.3\% (61) of participants said they did have experience with video game related motion detection devices while 18.7\% (14) said they had not. Of those who responded with having experience with motion detection devices, most participants stated that they were more familiar with using the Nintendo Wii Remote at 88.5\% (54), while quite encouragingly 52.5\% (32) said they had experience with the Microsoft Xbox Kinect and 14.8\% (9) had used the Sony Playstation Move before. Therefore, given by these statistics, it could be hypothesised that participants with or without prior experience with video game related motion detection devices could easily adapt to interacting with the Kinect balloon application.

This theory can be supported by the results taken from the post-questionnaire where it was shown that 93.0\% (66) of participants said they were able to intuitively push balloons around screen and 77.8\% (49) of participants said they were able to intuitively pop a balloon on screen.

An argument to support that the Kinect balloon application should permanently run on the  plasma screens of the Earl Mountbatten crush area after the project has come to an end can be derived from the results taken from both the pre-questionnaire and the post-questionnaire. Participants were asked how long they approximately spend in the crush area each day and of the 75 participants who responded, 68\% (51) of them stated that they approximately spend less than 20 minutes in the crush area each day, thus implying that most participants are generally only in the crush area for a short period of time. This is supported by the response given by the question where participants were asked to specify what they normally use the Earl Mountbatten crush area for. The response showed that 58.7\% (44) of participants typically use the crush area as a waiting area and 50.7\% (38) of participants are normally just passing through en route to some other location. However, given by the results of post-questionnaire, where 81.9\% (59) of participants stated they would like to play with the balloons again in the future and 70.8\% (51) of participants stated they would like to use the application again to view content that interests them, perhaps the Kinect balloon application would prove generally quite popular to students as something to quickly interact with in order to read various articles or to have some fun with while waiting between lectures.

It should be noted that the results of both the pre-experiment questionnaire and the post-experiment questionnaire in chart and table form can be found attached to this document, located in the appendices section.

\subsectionauthor{Utrecht Visit}{Lauren Hill \& John Truesdale}

After making the appropriate changes and additions to the application after the initial evaluation, a second evaluation was undertaken which involved students from Utrecht University who were visiting Heriot-Watt whilst on a tour of Europe.

This evaluation was undertaken by members introducing the application to the students on both screens and for another member to make observational notes on their behaviour. In addition, other members were present to discuss and listen to feedback from the students on what they liked and on what could be improved:

\begin{itemize}
	\item Users no longer struggled to understand what to do with the program and were immediately performing basic interactions such as moving and bursting balloons.
	\item Users immediately understood to hold their hand over the X in order to close the content window, and were neither frustrated by the time it took to complete the action, nor confused by accidentally using it.
	\item Users wished to interact with the airplane, by bashing balloons at it with the intention of destroying it.
	\item Users enjoyed multi-user interaction on each screen in addition to interaction with users on a separate screen. Users began their own game where they would try to hoard balloons from the other team's screen in addition to spamming the other screen with all the balloons.
	\item Few users wished to customize balloons, however those that did understood the required actions quickly and did not struggle to do so.
\end{itemize}

In addition, we also received several useful suggestions for future work from the participants:

\begin{itemize}
	\item More balloons were desired from the participants - particularly as they were playing their game in hoarding balloons on each screen.
	\item Users wished to be able to send a message to a twitter account that would then be displayed on a box held by a balloon. This balloon would not hold any content, and so when burst would not display a content box.
	\item In addition to sending a twitter message, users expressed their desire to do the same using Facebook: a user could post to a ``Heriot-Watt Balloons'' Facebook page -- a message that would then display on a balloon. If the user posted content such as a link or photo, this would then display as any regular content, where the balloon could be burst and a QR code would be generated. The image used would be the profile image of the poster.
	\item One user suggested that when bursting twitter balloons, a bird animation should fly out as if just being released.
	\item It was suggested that our QR code could be much simpler. As the code currently contains a lot of unneeded data it is very complex and the user must get close to the screen with their smartphone in order to scan it (also causing a problem of the user's detected hands covering up the code). If we removed this unneeded data the QR Code would be much simpler and therefore could be read from further away. In addition the hand images for users should not be drawn underneath the QR Code. 
\end{itemize}

Finally, a friendly interaction among the users themselves was noted, where users using one screen would shout to users on another screen to pass them balloons. This was a very pleasing observation, as it shows that the program does stimulate friendly engagement -- a concept we were keen to encourage with the program's location being the learning zone.

\clearpage{}

\subsectionauthor{Content Submission}{Alexander Macrae}
\newcommand{\webEvalQPic}[1] {
  \begin{figure}[h]
  \begin{centering}
  \includegraphics[width=\textwidth]{Diagrams/Website-EvalQ#1.png}
  \par\end{centering}

  \caption{Evaluation results (Question #1)}
  \label{WebEvalQ#1}
  \end{figure}
}

The website evaluation was done to examine how useful, usable and effective the website was. Usefulness is whether students feel that they have a need for this software. This is important because a significant part of the application is that the users can submit content. If the users find this useful they will use our application more.

Usability is how well participants can navigate the website and to carry out their goals with the website. If the website is very usable it will mean that users will find that they can quickly get what they want done without frustration or confusion. If the website is very usable then users will be more likely to come back and use it again.

Finally the effectiveness is analysed. The effectiveness is how well the website does the job it is supposed to do. Effectiveness is different from usability because effectiveness measures whether the website does what it is supposed to and usability measures whether users can do this easily. The job of the website in this case is to allow users to submit content to the balloon system in the aim of sharing content with others.

\subsubsection{How the website was evaluated}

To evaluate the website a questionnaire was created with which the participants could fill out after trying to submit a balloon with the website. The participants were usually people who have used the balloon news program but as the questionnaire was put online, some people who answered it may not have used it before. All participants had to have been members of the MACS network as this was needed for a login.

The questionnaire has 7 questions which cover the subjects above. It was made short as the other two questionnaires were quite long and it was decided not to take up too much time for the participant.

The questionnaire starts off by trying to asses the person who is answering the questionnaire. Have they used the balloon news program and do they share content. The first is a control question and the second is trying to evaluate the usefulness of this website. The next five questions are all on the website, asking the user how efficient and effective it was.

\clearpage{}

\subsubsection{Analysing the Results}

\subsubsection*{Question 1}
\webEvalQPic{1}
The results show that most people who filled out the questionnaire had used the balloon news before. This was due to most of the responses being after the participant had played around with the Balloon News game.

\subsubsection*{Question 2}
\webEvalQPic{2}
From this it can be seen that the number of people who share content online was high. However a third of the respondents did not share online content. As two thirds of the respondents answers yes to this question it is fair to assume that there is a market for allowing students to share content with each other and that the Balloon News program would be a good tool for that. 

\clearpage{}
\subsubsection*{Question 3}
\webEvalQPic{3}
This data shows that the purpose of the site was fulfilled in all cases. Every participant in out questionnaire who tried to submit an article succeeded.

\subsubsection*{Question 4}
\webEvalQPic{4}
From this multiple choice answer it can be seen that the website was found easy to use. No participants said that the website was difficult. However the participants were not as numerous at saying that it was very easy. 

\subsubsection*{Question 5}
\webEvalQPic{5}
The response from this question on design was not as high as for the ease of use. However it was not overly negative either which suggests that while participants were not in favour of the design, they did not dislike it to an extreme. This does suggest though that the design should be improved.

\subsubsection*{Question 6}
\webEvalQPic{6}
The response to this question gives an impression of how that targeted consumer of the application feels about its usefulness. From this it can be found that participants did think that the website would be useful as over three quarters of respondents said that it would be useful.

\subsubsection*{Question 7}
This was an open feedback question. The participants were asked whether they wanted any extra features. Most of the comments centred on the submissions of images with the post. One participant commented that ``It was annoying to have to submit an image''. Another suggested that a URL could be given instead. 

Other feedback was on the design. Participants didn't think that the design was brilliant and one person commented that it ``looks like it was done by a first year high school student!''.

One of the other extra features that were asked for was to be able to see where a submitted piece of content was on the screens. This could be considered for future development.

\clearpage{}
\subsubsection{Conclusion}

The original aim of the questionnaire was to assess the usability, usefulness and effectiveness of the website consumers will use to submit content to the screens. 

In terms of usability the results were very positive. All users found that they could submit an article, which indicates that the usability is not stopping users from completing their objectives. Participants also fed back that the website was generally easy to use with 56\% saying it was easy to use and 33\% saying it was very easy to use. This indicates very strongly that the usability of the website is high. One area which is not as positive is the design. A good design makes the usability easier. While the ease of use was rated as high the design was not rated as high. Therefore it could be that if the design was improved the usability could also be improved.

The second factor was usefulness. Usefulness is important as if the website is not useful then no one will use it and the content on the screens will not be as relevant to students. Usefulness was assessed through asking participants whether they shared content online and how useful they felt the website was. Sixty-seven per cent of respondents revealed that they share content online. If participants share content online then they are more likely to share content on our system as our system is also online and any social factors that may make someone not share content online with others will not apply to them as they already do it. Participants also responded that they thought the website would be useful with 78\% saying so.

The final factor being analysed was effectiveness. The goal of the website was to give students a way of submitting content to be displayed on the Balloon News screens. This was analysed through asking participants whether they could complete the objective. 100\% of participants manage to submit an article to the screens so the website is effective at what it is designed for. 

In summery the website was a success at allowing users to submit content to screens and participants thought it was a good idea. However the design could be given an overhaul and the process could probably be made slicker. An idea for the future may be to link the Balloon News website with Facebook or other online content sharing websites and allow users to add content to the balloon news feed by ``likin'' an article. This would take some research into implementations but if it could be done then it simplifies the process and it integrates the system into something users are already familiar with and use every day, thereby increasing the market as it were. 


\clearpage{}

\subsectionauthor{Future Work}{Lauren Hill \& John Truesdale}

After collaborating all of the data and feedback from user evaluation, in regards to both the results of the evaluation and observation of user behavior, a general consensus of where the application could be improved was agreed upon. The basic concept involved around the integration of a game to the current system but unfortunately due to time constraints was not achievable and therefore is detailed below so that future teams could potentially develop from.

The game would be a social application integrated within the current system that users would begin via a certain gesture or button. It would have the following game modes and would utilize both screens:

\begin{enumerate}
	\item Player vs Computer
	\item Player vs Player 
	\item 2 Players vs 1 Player
	\item 2 Players vs 2 Players
\end{enumerate}

In relation to the modes, it would look to incorporate a jump in, jump out feature where users could join and quit as they wished whilst a game was in play and thus have games start at random intervals. If found to be unfeasible it could require users to sign up in order to instigate games against the system or other players. In either case the game's duration would be short, from around 30 seconds to 1 minute.

Once commenced the game would randomly assign a large number of balloons, of varying colours, to both screens and have a very strong wind moving in a random, ever-changing direction. The interaction, and main goal, would be to gain points by completing certain commands or rounds, that is the same for both screens, that randomly appear on screen after a certain duration such as every 5 to 10 seconds. The rounds themselves would be set around getting rid or collecting a particular type of balloon, for example collecting only the red balloons and getting rid of all others. A variation to this could be to have as many balloons on the screen as you can or try to have none at all. The score would be calculated at the end of each round and a player would gain a point for every balloon on their screen that meets the condition and lose a point for not. In terms of the social interaction this would require the user to push and collect balloons on their screen whilst competing with the other player. The difficulty lies in dealing with the large number of fast balloons on your screen which the rounds will require you to constantly manage without much time to think.

This model was constructed entirely from user obervation and feedback and fits nicely within the desired user model for the application. By integrating this aspect into the current system it would allow for the application to address it's main negative feedback and allow for a mix of social interaction and data visualiation.
