In this section we describe what the Balloon Server does and how it fits in the
project. 

\subsection{Specifications}

The goal of the Balloon Server is to make users believe that, no matter how
many screens are used for display, there is only one Balloon application. One
way to achieve this is to make the screens look 'connected'. That is, after
pushing a balloon through the left edge of a screen, the balloon should
reappear at the right edge of another screen. Another way is to enforce that a
balloon is shown only one screen at any given time.

We want users to believe that the balloon that appeared in one screen is the
same balloon that disappeared in the other screen. Since the appearance of
balloons is customisable (the colour and texture can be changed by the user),
believing they are the same is easier if customisations are kept when
transferring the balloon from one screen to another. In addition, it is more
believable if balloons keep their speed and direction when moving across 
screens.

The server is responsible for notifying the client when balloons should appear
on its screen and how (e.g. what type of balloon, caption text, content
to show when popping the balloon). The server does not create the data needed to
display balloons itself. This task is the responsibility of the Web Feed. In
order to have balloons to provide to the client, the server has to regularily
retrieve the information needed to create these balloons from the feed.

Something to keep in mind is that the content of this feed changes with time
(i.e. new items are added from the Twitter aggregator and user-submitted 
content) but we want to keep the number of balloons on each screen at a
reasonable level. This is why the server has to decide which balloons should be
kept and which should be discarded when new items are retrieved from the feed.
We have decided to discard all balloons without a corresponding feed item.

An additional requirement is that the number of screens connected to the server
is not fixed and can change over the lifetime of the server execution. This
means that connecting and disconnecting screens from the server can be done at
any time and will not halt or impair the processing of balloons. For example, 
when a screen is added the server requests more items from the feed. These new
items are used to create new balloons to populate the new, empty screen. When 
disconnecting a screen, its balloons are transferred to the remaining screens.

Late in the project, we decided to introduce on-screen 'tips' to help new users
interact with our application. This was materialised as a screen carrying a label
showing the tip text behind it. The server is responsible for deciding when 
such planes are shown, transferring them between screens and deciding when they
have to disappear. Indeed we felt that always having a plane on a screen could
be annoying for the users, especially since planes can collide with balloons 
and push them away when users might be interacting with them.

\clearpage{}
\subsection{Design}

\subsubsection{Messaging}

\subsubsection{Persistance}

\subsubsection{Threading}

\subsection{Configuration}
