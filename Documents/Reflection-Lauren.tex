
While it goes without saying that I'm very proud of what the team accomplished in the end, I think the project's overall success was in how we got there. From a project management perspective, we remained organized, on schedule, had significant progress to show every week and in the end produced a piece of work exactly to the specification that was set at the start, as well as some valuable user feedback and prospects for the future.

We did miss one deliverable near the start of the project where the prototype was not fully tested before demonstration, however we very quickly learned from that mistake and adapted. We added an extra meeting time the day before our regular meetings with Judy to make sure everything would run smoothly for the demo the next day, and reduced programming deadlines to the Friday rather than the Sunday before, to give the lead programmer adequate time to make sure everything melded well together.

I feel if we had started the project on a different foot, by doing initial research to determine how users would want to use the Kinect, we would have come out with a completely different result. Feedback towards the end of the project indicated that a game would have fit very nicely in the environment the application was being used in, however it remains unknown if we would have got this feedback from a simple questionnaire. If I was to make a suggestion to groups who wished to do something similar, I would say to use the framework we've created already and design a game around it -- perhaps still integrated with the information push concept.

I do hope the application is used in the crush area for its intended purpose. I am proud that we produced a polished, bug-free complete piece of work and on top of that learned valuable lessons in time management and teamwork.
