The application loads values from a configuration file at start-up, and a default file is created if not file is found.
The valid options in the configuration file are show below.

\begin{tabular}{|p{5.0cm}|p{3.0cm}|p{7.6cm}|}
\hline \multicolumn{3}{|c|}{Common (shared) values}  \\ \hline
Value name & Default value & Comments \\ \hline

LogToConsole & true & Whether log messages are displayed in the console (`stdout') or not.  \\ \hline

LogToFile & false & Whether log messages are written to a file or not. \\ \hline

LogFilePath & null & Path of the file log messages are written to (only if `LogToFile' is true). \\ \hline

InputType & `Mouse' & Which input device to use (`Mouse' or `Kinect'). \\ \hline

SerializerType & `Binary' & Which serialiser to use for network communications (`Binary' or `Text'). \\ \hline

LogNetworkMessages & false & Whether incoming/outgoing messages are logged or not. \\ \hline

BalloonDeadzoneMultiplier & 1.1 & Size (proportional to the size of a balloon) of the deadzone at each side of the screen. Balloons have to move past this zone before changing screens. \\ \hline

\end{tabular}

\clearpage{}

\begin{tabular}{|p{5.0cm}|p{3.0cm}|p{7.6cm}|}

\hline \multicolumn{3}{|c|}{Server specific values} \\ \hline

Value name & Default value & Comments \\ \hline

LocalIPAddress & 0.0.0.0 & IP address of the interface the server uses to listen for incoming connections. The default value means `any interface'. \\ \hline

LocalPort & 4000 & Port used to listen for incoming connections. \\ \hline

FeedURL & \emph{See note below} & URL of the Web Feed API. The \verb${0}$ part gets replaced by the number of items to request. \\ \hline

FeedTimeout & 60000 & Interval of time, in seconds, after which the server should request new items from the feed. \\ \hline

MinBalloonsPerScreen & 1 & Threshold below which the server requests new items from the feed. \\ \hline

MaxBalloonsPerScreen & 5 & Number of feed items (per screen) to request from the feed. \\ \hline

BalloonVelocityLeftX & -10.0 & Initial velocity of a balloon appearing on the right side of the screen (X). \\ \hline

BalloonVelocityLeftY & 0.0 & Initial velocity of a balloon appearing on the right side of the screen (Y). \\ \hline

BalloonVelocityRightX & 10.0 & Initial velocity of a balloon appearing on the left side of the screen (X). \\ \hline

BalloonVelocityRightY & 0.0 & Initial velocity of a balloon appearing on the left side of the screen (Y). \\ \hline

PlaneVelocityLeftX & -2.0 & Initial velocity of a plane going from right to left (X). \\ \hline

PlaneVelocityLeftY & 0.0 & Initial velocity of a plane going from right to left (Y). \\ \hline

PlaneVelocityRightX & 2.0 & Initial velocity of a plane going from left to right (X). \\ \hline

PlaneVelocityRightY & 0.0 & Initial velocity of a plane going from left to right (Y). \\ \hline

PlaneInitialY & 0.5 & Initial, normalised value of a plane's initial Y coordinate on the screen. \\ \hline

\end{tabular}

\paragraph{Note}

The default URL of the Web Feed is:\\ \url{http://www.macs.hw.ac.uk/~cgw4/balloons/index.php/api/getFeed/{0}}

\clearpage{}

\begin{tabular}{|>{\raggedright}p{5cm}|>{\raggedright}p{3.6cm}|>{\raggedright}p{7cm}|}

\hline
\multicolumn{3}{|c|}{Client specific values}\tabularnewline
\hline

Value name & Default value & Comments \tabularnewline
\hline

FullScreen & false & comment
\tabularnewline\hline

ScreenWidth & 1366 & comment
\tabularnewline\hline

ScreenHeight & 768 & comment
\tabularnewline\hline

MessageDisplayTime & 30000 & comment
\tabularnewline\hline

PopAnimationTime & 2500 & comment
\tabularnewline\hline

PopAnimationEnabled & true & comment
\tabularnewline\hline

PopAnimationAlpha & 0.3 & comment
\tabularnewline\hline

PopAnimationBeta & 5.0 & comment
\tabularnewline\hline

PopAnimationScale & 2.0 & comment
\tabularnewline\hline

UseHtmlRendering & true & comment
\tabularnewline\hline

RemoteIPAddress & localhost & comment
\tabularnewline\hline

RemotePort & 4000 & comment
\tabularnewline\hline

EnableHighFive & false & High five allows two users to pop a balloon by each using a hand to create a clap (i.e. a high five). We decided to disable this because it's actually very hard to do!
\tabularnewline\hline

KinectMovementThreshold & 2.0 & This is how fast both hands need to be moving for a clap to be triggered. It uses the velocity from the physics object attached to the hands, so the value is in metres/second.
\tabularnewline\hline

KinectMaxHandRange & 2.0 & This is how far the second hand can be from a balloon to trigger a clap, when the first hand collides with the balloon. Measured in internal metres.
\tabularnewline\hline

KinectMinAttackAngle & 0.4 & This value ranges from -1 to 1 and is the cosine of the difference in angle allowed between hands and the balloon for triggering a clap. A value of 1 indicates the hand moves move directly at the balloon, 0 allows the hand to be moving parallel to the balloon (90 degrees) and -1 allows the hand to be moving directly away from the balloon.
\tabularnewline\hline

KinectXSensitivity & 1.5 & This is the horizontal multiplier for the Kinect input. Should be adjusted for each environment to maximize usability.
\tabularnewline\hline

KinectYSensitivity & 2.0 & This is the vertical multiplier for the Kinect input. Should be adjusted for each environment to maximize usability. Generally this should be bigger than the horizontal multiplier.
\tabularnewline\hline



\end{tabular}
