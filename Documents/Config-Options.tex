The application loads values from a configuration file at start-up, and a default file is created if not file is found.
The valid options in the configuration file are show below.

\begin{tabular}{|>{\raggedright}p{5cm}|>{\raggedright}p{3.6cm}|>{\raggedright}p{7cm}|}

\hline
\multicolumn{3}{|c|}{Common (shared) values}\tabularnewline
\hline

Value name & Default value & Comments \tabularnewline
\hline

LogToConsole & true & comment
\tabularnewline\hline

LogToFile & false & comment
\tabularnewline\hline

LogFilePath & null & comment
\tabularnewline\hline

InputType & Mouse & comment
\tabularnewline\hline

SerializerType & Binary & comment
\tabularnewline\hline

LogNetworkMessages & false & comment
\tabularnewline\hline

BalloonDeadzoneMultiplier & 1.1 & comment
\tabularnewline\hline


\end{tabular}
\begin{tabular}{|>{\raggedright}p{5cm}|>{\raggedright}p{3.6cm}|>{\raggedright}p{7cm}|}

\hline
\multicolumn{3}{|c|}{Client specific values}\tabularnewline
\hline

Value name & Default value & Comments \tabularnewline
\hline

FullScreen & false & comment
\tabularnewline\hline

ScreenWidth & 1366 & comment
\tabularnewline\hline

ScreenHeight & 768 & comment
\tabularnewline\hline

MessageDisplayTime & 30000 & comment
\tabularnewline\hline

PopAnimationTime & 2500 & comment
\tabularnewline\hline

PopAnimationEnabled & true & comment
\tabularnewline\hline

PopAnimationAlpha & 0.3 & comment
\tabularnewline\hline

PopAnimationBeta & 5.0 & comment
\tabularnewline\hline

PopAnimationScale & 2.0 & comment
\tabularnewline\hline

UseHtmlRendering & true & comment
\tabularnewline\hline

RemoteIPAddress & localhost & comment
\tabularnewline\hline

RemotePort & 4000 & comment
\tabularnewline\hline

EnableHighFive & false & High five allows two users to pop a balloon by each using a hand to create a clap (i.e. a high five). We decided to disable this because it's actually very hard to do!
\tabularnewline\hline

KinectMovementThreshold & 2.0 & This is how fast both hands need to be moving for a clap to be triggered. It uses the velocity from the physics object attached to the hands, so the value is in metres/second.
\tabularnewline\hline

KinectMaxHandRange & 2.0 & This is how far the second hand can be from a balloon to trigger a clap, when the first hand collides with the balloon. Measured in internal metres.
\tabularnewline\hline

KinectMinAttackAngle & 0.4 & This value ranges from -1 to 1 and is the cosine of the difference in angle allowed between hands and the balloon for triggering a clap. A value of 1 indicates the hand moves move directly at the balloon, 0 allows the hand to be moving parallel to the balloon (90 degrees) and -1 allows the hand to be moving directly away from the balloon.
\tabularnewline\hline

KinectXSensitivity & 1.5 & This is the horizontal multiplier for the Kinect input. Should be adjusted for each environment to maximize usability.
\tabularnewline\hline

KinectYSensitivity & 2.0 & This is the vertical multiplier for the Kinect input. Should be adjusted for each environment to maximize usability. Generally this should be bigger than the horizontal multiplier.
\tabularnewline\hline


\end{tabular}
\begin{tabular}{|>{\raggedright}p{5cm}|>{\raggedright}p{3.6cm}|>{\raggedright}p{7cm}|}

\hline
\multicolumn{3}{|c|}{Server specific values}\tabularnewline
\hline

Value name & Default value & Comments \tabularnewline
\hline

LocalIPAddress & 0.0.0.0 & comment
\tabularnewline\hline

LocalPort & 4000 & comment
\tabularnewline\hline

FeedURL & 'http://www.macs.hw.ac.uk/~cgw4/balloons/index.php/api/getFeed/{0}' & comment
\tabularnewline\hline

FeedTimeout & 60000 & comment
\tabularnewline\hline

MinBalloonsPerScreen & 1 & comment
\tabularnewline\hline

MaxBalloonsPerScreen & 5 & comment
\tabularnewline\hline

BalloonVelocityLeftX & -10.0 & comment
\tabularnewline\hline

BalloonVelocityLeftY & 0.0 & comment
\tabularnewline\hline

BalloonVelocityRightX & 10.0 & comment
\tabularnewline\hline

BalloonVelocityRightY & 0.0 & comment
\tabularnewline\hline

PlaneVelocityLeftX & -2.0 & comment
\tabularnewline\hline

PlaneVelocityLeftY & 0.0 & comment
\tabularnewline\hline

PlaneVelocityRightX & 2.0 & comment
\tabularnewline\hline

PlaneVelocityRightY & 0.0 & comment
\tabularnewline\hline

PlaneInitialY & 0.5 & comment
\tabularnewline\hline

\end{tabular}
