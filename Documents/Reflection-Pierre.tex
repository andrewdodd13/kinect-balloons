My feeling is that this project was overall both very successful and satisfying.
Together, we have created an interactive `news board-like' application that doubles
as an entertaining mini-game using the Kinect. In addition, I believe we have 
maintained a level of quality high enough for this project to be re-used as the 
base for a small-scale, distributed and interactive 2D game. Also, the different
parts in the project are clearly separated which suits a team of several people
working in parallel. I think doing this with a team of nine in twelve weeks' 
time is an achievement we can be proud of. Of course, as with any project there 
also are aspects we could have done differently or spent more time on and done better.

One of these aspects is the `direction' of the project, that is, finding a reason
why users would want to play and interact with it. Our direction was half-way
between a `news' application and a game. While we have seen that users enjoyed
trying it out we also observed that their interest in our application vanished
quickly. In my opinion we did not anticipate what users would find fun and what
would make them want to play with it. I regret that we did not choose to include
a stronger gaming aspect, but keep in mind this is easy to say in hindsight.

Indeed, when we brainstormed for ideas at the beginning of this project I think
nobody knew for sure how an ambitious project like this one would work out. I
remember that choosing an idea was a difficult part of the project, but I also
believe we chose something that we all agreed on and were confident we could
deliver. Both `news' and 'game' ideas were proposed and in the end we went for
a combination of both aspects.

This choice could have confused us and made it easy to be distracted from our goals.
This did not happen however, in large part thanks to John, Lauren and Andrew's
management of the project as well as everyone working well in teams and collaborating
between teams. I think we were dedicated and focused on delivering good quality work
throughout the weeks and that was an important factor in making this project work.

Another factor that I see as important was that we managed to create a working
prototype of all parts of the application very early in the life of the project.
I felt that the first interactions between the different parts of the application 
was very rewarding. It might sound silly but I was thrilled when we first could
move balloons between screens using the Kinect!

Integrating these parts was not easy and indeed this showed during one of our
early demonstrations which did not go so well. However, after this we focused
on eliminating bugs that caused this mishap, improving the quality of the code and making
it more resilient to errors. I believe that after this was done the hardest
(technically) was behind us; adding features such that improving the text layout
and general appearance became much easier. 

I had never worked on a Kinect project or such a public-facing project and this
added an interesting new twist for me. This gave me much motivation to work on
several different technical aspects in this project (writing the server and
messaging code with Brieuc, trying out and making HTML rendering work in the 
client or adding some on-screen animations). Indeed I really enjoyed the technical
work I have done as well as the collaboration and task allocation within this 
project. I was glad to see how everyone used their different strengths to make
sure that everything was taken care of. 

If we had more time, I would definitely be keen to steer this project in a more
`game-like' direction. It was nice to see users become enthusiastic when 
improvising `mini-games' such as trying to keep all of the balloons off each
player's screen. I do hope that next year students will be able to pick this 
project up from where we are leaving it and improve it in very fun ways! I also
wish them to be part and go through such an interesting and satisfying project
such as this MEng final year project.
