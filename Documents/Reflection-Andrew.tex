For my part of the project, I mainly worked on the client application's design
and application code. I was also in general charge of maintaining the the Git
repository and making sure the coders met their deadlines. Throughout the
project I cannot think of an instance where any part of the project was
delivered late; there were some issues with a broken prototype one week but
this was more to do with a lack of structure within the group at that point in
time. After we rectified this issue, there were no major hiccups with the
development process in general.

In terms of our development process, using Git proved to be very effective, even
though a few members of the group initially had difficulty using it. It allowed
me as a reviewer to clearly see which changes had been performed by whom and
also allowed us to revert broken changes or choose a ``frozen'' build for 
demostration purposes. Managing several discrete branches proved to be slightly
annoying but fortunately I was the only one in the group who really had to deal
with this.

Our choice of XNA and C\# was the correct one, in my opinion. Using C\# made
using the official Kinect SDK very easy as many of the samples were written in
it and it is one of Microsoft's recommended languages to use for it. The use of
XNA drastically reduced our development time as it provides a lot of features
for managing static resources (images) and drawing to the screen. It also
defines quite a rigid structure for simple games which removes some of the pain
of doing that manually.

One library that I was not entirely keen on was the Farseer Physics library. This
library calculates the positions of the physical bodies (balloons, etc.) very
realistically, however it is this realism that makes it slightly clunky to use.
This is especially problematic when combined with the hands from the Kinect --
the user appeared to experience a feeling of input lag. This is caused by the
physics system moving the hands to their position gradually. We tried turning
the sensitivity way up but this didn't really improve the situation much; 
perhaps an alternative method could have improved this further.

With client itself, I feel the design is very clean. As much as is possible, we
have tried to split the system into distinct components; with each component
being coded to an interface allowing it to be easily swapped with a different
implementation. This should make it extremely easy for another team to pick up
the project and understand how all the components interact with each other. It
also made it easier for multiple people to work concurrently on the client code,
although there were never more than two people working on the client in 
practise.

Furthermore, the code is fairly well documented -- the interfaces in particular
-- although some of the implementations could do with a bit more commenting. 
This, however, shouldn't present a problem because the implementation should be
doing exactly what is specified on the interface.

The design of the client is partly dictated by the messaging model of the
entire system. Whilst I would have opted for a third-party remote-procedure-call
(RPC) model for network communications, the two coders on the networking team
chose to write their own implementation. Whilst this worked very well for our
team, I feel it may present a barrier to another group trying to continue work
on the system as they may have to learn how this part of the system behaves.

Overall, I think that the client system is an excellent product to have reached
for this project. It appears to have been well received by users and is a well-
designed piece of software. Many of the issues users raised during testing were
fixed before the final product was handed in, but even these were relatively
minor in the scope of the entire project.
